%
\section{Theoretical Background}\label{sec:theoretical background}
%
\subsection {Cloud Computing}
Cloud computing has transcended its initial emergence decades ago to become a foundational pillar of modern technological advancement. It has brought revolution by ensuring flexibility, scalability, and cost-effectiveness. Through the ingenious power of virtualization, cloud computing has pioneered a model where both software and hardware resources are delivered on demand, precisely mirroring the user's requirements \cite{Chou15}. The cloud's inherent global reach enables seamless access to applications and files from any device, anywhere in the world, fostering enhanced collaboration and productivity\cite{NIST}.\\\\
Cloud computing offers diverse service models to cater to a range of technical capabilities and project demands. 
\textbf {\ac{SaaS}}: SaaS offers readily available, pre-configured software applications accessible through a web browser \cite{Odu18,HBS21}. 
\textbf {\ac{PaaS}}: PaaS offers a development environment with programming tools and resources to the users where they can leverage this platform to build custom applications without managing the underlying infrastructure \cite{Odu18,HBS21}. 
\textbf {\ac{IaaS}}: IaaS grants users fine-grained control over computing resources such as servers, storage, and network components to build applications \cite{Odu18,HBS21}. \\

\subsection {Serverless Computing}

The relentless evolution of cloud computing ushers in a new era with the captivating emergence of serverless computing. This novel paradigm fundamentally reshapes the existing model by shifting the focus from infrastructure management to code execution. This platform abstracts away the underlying server complexities, empowering developers to concentrate on crafting the business logic of their applications \cite{Baldini17, SA20}.\\\\
serverless computing offers a cost-effective, scalable, and elastic platform. It offers the scaling of resources that adapts to the fluctuating demands without any manual intervention. This results in cost savings as the developers only pay for the resources the application requires \cite{CIM+19}.\\