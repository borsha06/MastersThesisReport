%
\section{Theoretical Background}\label{sec:theoretical background}
%
\subsection {Cloud Computing}
Cloud computing has transcended its initial emergence decades ago to become a foundational pillar of modern technological advancement. It has brought revolution by ensuring flexibility, scalability, and cost-effectiveness. Through the ingenious power of virtualization, cloud computing has pioneered a model where both software and hardware resources are delivered on demand, precisely mirroring the user's requirements \cite{Chou15}. The cloud's inherent global reach enables seamless access to applications and files from any device, anywhere in the world, fostering enhanced collaboration and productivity\cite{NIST}.\\\\
Cloud computing offers diverse service models to cater to a range of technical capabilities and project demands. 
\textbf {\ac{SaaS}}: SaaS offers readily available, pre-configured software applications accessible through a web browser (\cite{Odu18},\cite{ HBS21}). 
\textbf {\ac{PaaS}}: PaaS offers a development environment with programming tools and resources to the users where they can leverage this platform to build custom applications without managing the underlying infrastructure (\cite{Odu18},\cite{ HBS21}). 
\textbf {\ac{IaaS}}: IaaS grants users fine-grained control over computing resources such as servers, storage, and network components to build applications (\cite{Odu18},\cite{ HBS21}). \\

\subsection {Serverless Computing}
